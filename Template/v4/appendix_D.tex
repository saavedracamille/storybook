%%%%%%%%%%%%%%%%%%%%%%%%%%%%%%%%%%%%%%%%%%%%%%%%%%%%%%%%%%%%%%%%%%%%%%%%%%%%%%%%%%%%%%%%%%%%%%%%%%%%%%
%
%   Filename    : appendix_C.tex 
%
%   Description : This file contains the interview transcript of Ms. Pacis.
%                 
%%%%%%%%%%%%%%%%%%%%%%%%%%%%%%%%%%%%%%%%%%%%%%%%%%%%%%%%%%%%%%%%%%%%%%%%%%%%%%%%%%%%%%%%%%%%%%%%%%%%%%

\chapter{Ms. Maria Clara Pacis Interview Transcript}
\label{sec:appendixc}

Date: October 12, 2016 \\
Time: 11:20am-11:32am \\
Interviewer: Camille Saavedra \\
 
Introduction of Thesis Topic before actual interview
 
Camille: What are the common types of stories?
 
Ms. Pacis: In Facebook?
 
Camille: Yes, or in general.
 
Ms. Pacis: I’m not a Facebook user. What do you mean? Life stories, autobiography, profiles. Those are the usual. But maybe for Facebook, it would just be profiles. Meaning, not birth to death, just a particular moment, event, time, or a time period.
 
Camille: What are the common elements of stories used in writing life stories?
 
Ms. Pacis: All the elements of fiction. You can use. Plot, character, dialog…
 
Camille: Are all of these always required?
 
Ms. Pacis: Yes. We don't have maybe that dialog. Definitely for your characters to be more alive, they will have to be speaking.
 
Camille: What would you consider to be the essential parts of a story?
 
Ms. Pacis: All of that. It also depends on what you want to focus on. So for example, okay, do you already have an idea who you want to target for your stories?
 
Camille: Just any Facebook user.
 
Ms. Pacis: No, you have to choose. It cannot just be any Facebook user. If all the Facebook users says I woke up this morning and brushed my teeth. I mean, you definitely won’t have a story. But let's say you choose Kim Kardashian, that's different right? What of Kim Kardashian is so fascinating? Maybe her obsession with her body or something like that? That's what you focus on. Okay? So, you have to look for a subject that is interesting. Or has something to say. Or have done something interesting. I would encourage you to choose themes or subject matter or people who have done good things. Very productive to help others. How many are you going to do ba?
 
Camille: We're not sure yet.
 
Ms. Pacis: Well, you can have a person like Kim Kardashian and contrast with someone who does good. So it's obvious to the reader who you should be emulating.
 
Camille: What is the distinction between a regular story and an autobiography?
 
Ms. Pacis: A life story is based on fact. Regular stories, there are fiction which is imagine or made up.
 
Camille: What are some common structures of stories that you notice?

Ms. Pacis: Did you guys take HUMALIT? You are asking me very basic questions. 

Ms. Pacis: There's beginning, middle, end, And well… it depends on how you guys want to write your stories. If you want to write stories and make it sound fiction, then you got to have a conflict. There should be a problem. Or it could just be a straight up narrative.
 
Camille: Have you written stories that target teenager or adult readers?
 
Ms. Pacis: Many. Yeah.
 
Camille: What are some of these stories you've written?
 
Ms. Pacis: Just look for it na lang at National Book Store. But it’s more for the younger not college. 12 year olds.
 
Camille: How do you keep the audience motivated to finish reading the story?
 
Ms. Pacis: You just have to keep the story exciting. But if you're using Facebook stories for your data then not very long.
 
Camille: How would you know if a story is good or how would you evaluate a story?
 
Ms. Pacis: If it's well written, if it’s captivating, or if it keeps you interested. 