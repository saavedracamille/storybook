%%%%%%%%%%%%%%%%%%%%%%%%%%%%%%%%%%%%%%%%%%%%%%%%%%%%%%%%%%%%%%%%%%%%%%%%%%%%%%%%%%%%%%%%%%%%%%%%%%%%%%
%
%   Filename    : abstract.tex 
%
%   Description : This file will contain your abstract.
%                 
%%%%%%%%%%%%%%%%%%%%%%%%%%%%%%%%%%%%%%%%%%%%%%%%%%%%%%%%%%%%%%%%%%%%%%%%%%%%%%%%%%%%%%%%%%%%%%%%%%%%%%

\begin{abstract}
Nowadays, people use social media, and in particular, Facebook, to share information about themselves, their daily activities and the things that interest them. However, a Facebook user's data (posts, personal info, etc.) by itself does not provide a \textit{concise} narrative of events that can be considered a life story. 

This research focuses on automatically generating a life story from a person's Facebook text posts. It is able to extract user data from Facebook, process them, organize them according to their content, and utilize them to generate a person's life story.

The research showed the difficulty of working with user-generated data, because it can be brief, informal, and very context-sensitive. Therefore, the event classifier and story generator is dependent on the quality and sufficiency of user posts. Regardless, this research shows that Facebook can provide enough data to generate complete life stories about people.

\begin{flushleft}
\begin{tabular}{lp{4.25in}}
\hspace{-0.5em}\textbf{Keywords:}\hspace{0.25em} &Social media, Text understanding, Post classification, Life event detection, Storytelling, Story generation \\
\end{tabular}
\end{flushleft}
\end{abstract}
