%%%%%%%%%%%%%%%%%%%%%%%%%%%%%%%%%%%%%%%%%%%%%%%%%%%%%%%%%%%%%%%%%%%%%%%%%%%%%%%%%%%%%%%%%%%%%%%%%%%%%%
%
%   Filename    : appendix_B.tex 
%
%   Description : This file contains the interview transcript of Mr. Gojo-Cruz.
%                 
%%%%%%%%%%%%%%%%%%%%%%%%%%%%%%%%%%%%%%%%%%%%%%%%%%%%%%%%%%%%%%%%%%%%%%%%%%%%%%%%%%%%%%%%%%%%%%%%%%%%%%

\chapter{Mr. Genaro Gojo-Cruz Interview Transcript}
\label{sec:appendixb}

Date: September 29, 2016 \\
Time: 2:30pm-2:50pm \\
Interviewer: Janica Mae Lam \& Robee Khyra Mae Te \\

Introduction of Thesis Topic before actual interview

Robee: Ano yung mga common types of stories na usually ginagamit or sinusulat

Mr. Gojo-Cruz: Nagiiba iba eh.. depende sa mga issues ngayon, diba? Usually mga personal essays na maiikli na pang, ano lang, pang status kasi sinasabing wala nang time magbasa ang mga tao ngayon ng mga mahahaba. So yung status nya maiikli lang tapos depende yung theme usually kapag kung anong yung uso na topic, yun yung nagiging theme ng topic ng post.

Robee: ano yung mga common naman na elements? Kasi po focus po namin is life stories so gusto po namin malaman yung mga elements needed.

Mr. Gojo-Cruz: So tawag diyan, ano, personal essay or creative non-fiction. Yan yung tunguhin pala ninyo, ang tawag diyan ay creative non-fiction so ito yung batay sa totoong pangyayari, batay sa totoong experiences. Kung ito ay batay sa totoong pangyayari, laging ``I'', the one narrating my own story, so they use ``I'' as a pronoun. Your the subject mismo.

Janica: Sa mga creative non-fiction or personal essay may mga required ba na elements? Like akilangan ba may plot palagi?

Mr. Gojo-Cruz: Wala, wala naman. Basta 100… hindi naman 100.. Kung magagawang 100\% na kwento mo ng totoo yung akranasan yun yung creative non-fiction

Janica: So more on events yung creative non-fiction?

Mr. Gojo-Cruz: Kasama dun pero more on personal experiences na hindi lahat nakakadanas kaya  ikwenekwento mo kasi... kakaiba. Kwinekwento mo kasi hindi siya common. 

Janica: So different po siya sa biography?

Mr. Gojo-Cruz: Ang biography kasi nakafocus sa achievements kasi dito pwede mo ikwento yung, halimbawa, nagkaSTD ka kung pano ka naghanap ng ostiptal na magtretreat sayo na hindi ka kilala yung mga ganon, yung personal essay. So sa biography kasi puro highlights lang eh, diba? Ng buhay mo kwinenkwento mo mga rugs to riches, mga ganun, sa non-fic hindi ganun.

Robee: May mga common na structure ba yung mga elements, kunyai eto kailangan palaging nafofollow after yung ganitong element.

Mr. Gojo-Cruz: Ang nabubuti sa ano creative non fiction malaya ka walang structure. Bsta sa huli lagi kang may iiiwan sa reader mo.

Janica: Like parang conclusion?

Mr. Gojo-Cruz: Conclusion pero pag conclusion kasi amsyadong academic ang dating eh. Ano iiiwan mo sa reader anong point of view anong philosophy, anong pananaw, realization, anong reflection na hindi laging aral.

Robee: Pano mo magagawang  parang mamotivate yung audience niyo para matapos yung story?

Mr. Gojo-Cruz: Ako ang technique ko ano eh, pagnagsasalita ako sa essay ko hindi ako perpekto. Like as it is kwekwento ko, kung nakakahiya, wala akong paki. Kasi layunin mo talaga ay ma-ishare yung karanasan mo , kakahiya man ito or magmumukha kang sabaw, kasi ang problema sa mga writers, na ano, parang ang talitalino nila, napakaperpekto nila, so ang aking pagsulat hindi ganun. Laging may loop holes, minsan palpak kasi kapag nagkakamali ka mas nagiging human ka sa iyong readers, hindi laging ikaw matalino ikaw laging may alam. 

Janica: may pre process po ba kayo kunyari before magsulat may ginagawa po ba kayo or finoformulate?

Mr. Gojo-Cruz: Depende eh minsan may mga essays na pagdating ko dito sulat na kaagad kasi nadaanan ko mga gnun. Nadaananko sa taft mga ganun pero minsan halos wala naman ako nasusulat so hindi ko naman pinipilit. Kung walan gisusulat edi walang problema

Robee: Nakapagsulat na ba kayo gn stories para sa adults or teenagers?

Mr. Gojo-Cruz: Meron akong isang libro na Connect the dots, young adults ko. Collection of creative non fiction.

Robee: For our last question, pano niyo nirereview yung stories niyo, pano mo nalalaman na yung naproduce is okay na, naintindihan na ng reader mo or enough na?

Mr. Gojo-Cruz: Sa ano FB status, kung ipopost mo ito. Syempre mababa lang naman batayan nito kasi madali lang naman ilike ito. Syempre yung ano, yung number of naglike, ang pinakamaganda sa akin yung ano eh, yung number ng shares. Kasi kapag nagshare sila, lalagyan nila ng sarili nilang words kung paano nakapekto sa kanila yung story so yun yung number of nagshare nung post.
