%%%%%%%%%%%%%%%%%%%%%%%%%%%%%%%%%%%%%%%%%%%%%%%%%%%%%%%%%%%%%%%%%%%%%%%%%%%%%%%%%%%%%%%%%%%%%%%%%%%%%%
%
%   Filename    : abstract.tex 
%
%   Description : This file will contain your abstract.
%                 
%%%%%%%%%%%%%%%%%%%%%%%%%%%%%%%%%%%%%%%%%%%%%%%%%%%%%%%%%%%%%%%%%%%%%%%%%%%%%%%%%%%%%%%%%%%%%%%%%%%%%%

\begin{abstract}
People love sharing stories with one another because stories provide them with enjoyable experiences as well as help them learn new things. Nowadays, people use social media, and in particular, Facebook, to share information about themselves, their daily activities and the things that interest them. However, a Facebook user’s data (posts, photos, personal info, etc.) by itself does not provide a \textit{concise} narrative of events that can be used to adequately tell a person’s life story. 

There is no current work implemented which creates a textual story from a given set of user-generated social media data. This research focuses on automatically generating a life story from a person's Facebook text posts. To this end, a system called FB Stories was developed. It extracts user data from Facebook (with their permission), processes them, organizes them into different types according to their text content, and utilizes this knowledge in order to generate a person’s life story. 

The development of the system showed the difficulty of working with noisy user-generated data. The testing showed that, while there were issues with low precision and high recall, the system addresses the research gap and satisfies all of its objectives. 

\begin{flushleft}
\begin{tabular}{lp{4.25in}}
\hspace{-0.5em}\textbf{Keywords:}\hspace{0.25em} &Social media, Facebook, Text understanding, Post classification, Life event detection, Storytelling, Story generation \\
\end{tabular}
\end{flushleft}
\end{abstract}
